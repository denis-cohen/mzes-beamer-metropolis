% !TeX program = lualatex
\documentclass[10pt, aspectratio=169]{beamer}
\usepackage{pgfpages}
\usepackage{fontawesome}
\usepackage[english]{babel}
\usepackage[utf8]{inputenc}
\usepackage{graphics}
\usepackage[T1]{fontenc}

% Bibliography (choose backend according to your preferences)
\usepackage[authordate, natbib, useprefix=true, isbn=false,url=false, doi=false, backend=biber]{biblatex-chicago}  
\bibliography{../../../../library.bib}
\setbeamertemplate{bibliography item}[text]

% Size of bibliography entries
\renewcommand*{\bibfont}{\footnotesize}

% new citep command
\usepackage{fixltx2e}
\newcommand{\cpsh}[2][]{\textsuperscript{\fontsize{6}{6}\textcolor{mzescyan}{[{#1}\citealp[]{#2}]}}}
\newcommand{\cpsl}[2][]{\textsubscript{\fontsize{6}{6}\textcolor{mzescyan}{[{#1}\citealp[]{#2}]}}}
\newcommand{\csh}[2][]{\textsuperscript{\fontsize{6}{6}\textcolor{mzescyan}{[{#1}\citealp[]{#2};}}}
\newcommand{\csl}[2]{\textsubscript{\fontsize{6}{6}\textcolor{mzescyan}{\hphantom{[{#2}}\citealp{#1}]}}}
\newcommand{\cpshl}[3][]{\textcolor{mzescyan}{\rlap{\csh[#1]{#2}}{\csl{#3}{#1}}}}
\newcommand{\cpsm}[2][]{\textcolor{mzescyan}{\fontsize{6}{6} [{#1}\citealp[]{#2}]}}

% new color emphasize commands
\newcommand{\cemph}[1]{\textcolor{mzescyan}{#1}}
\newcommand{\gemph}[1]{\textcolor{mzesgold}{#1}}

% Layout
\mode<presentation>
{
	\usetheme[progressbar=frametitle]{metropolis}
	\useoutertheme[]{metropolis}
	\useinnertheme[]{circles}
	
	
	%remove navigation symbols
	\setbeamertemplate{navigation symbols}{}
	
	% define colors
	\definecolor{mzescyan}{RGB}{0,118,150}
	\definecolor{mzesgold}{RGB}{209,163,84}
	\definecolor{mzesdarkgold}{RGB}{127, 110, 87}
	\definecolor{mzesbg}{RGB}{255,255,255} % white
	% \definecolor{mzesbg}{RGB}{252,253,254} % grayish
	
	% Blocks
	\setbeamercolor{block body}{bg=mzescyan!10, fg=black}
	\setbeamercolor{block body alerted}{bg=normal text.bg!90!black}
	\setbeamercolor{block body example}{bg=normal text.bg!90!black}
	\setbeamercolor{block title alerted}{use={normal text,alerted text},fg=mzesgold,bg=black}
	\setbeamercolor{block title}{bg=mzescyan, fg = mzesbg}
	\setbeamercolor{block title example}{use={normal text},fg=example text.fg!75!normal text.fg,bg=normal text.bg!75!black}
	
	% Frames
	\setbeamercolor{fine separation line}{fg=mzesgold}
	\setbeamercolor{item projected}{fg=mzesbg}
	
	% Miniframes
	\setbeamercolor{section in head/foot}{fg=mzescyan,bg=mzesbg}
	
	% Title
	\setbeamercolor{background}{bg=mzesbg}
	\setbeamercolor{background canvas}{bg=mzesbg}
	\setbeamercolor{title}{fg=mzescyan, bg = mzescyan!5}
	\setbeamercolor{titlelike}{fg=mzescyan}
	\setbeamercolor{subsection in head/foot}{bg=mzesgold!10, fg = mzescyan}
	\setbeamercolor{author in head/foot}{bg=mzescyan, fg = mzesbg}
	\setbeamercolor{title in head/foot}{bg=mzescyan!5, fg = mzescyan}
	\setbeamercolor{date in head/foot}{bg=mzesbg, fg = mzescyan}
	\setbeamercolor{page in head/foot}{bg=mzesbg, fg = mzesgold}
	\setbeamercolor{frametitle}{fg=mzescyan, bg = mzesbg}
	\setbeamercolor{item}{fg=mzescyan}
	\setbeamercolor{normal text}{bg=mzesbg,fg=black}
	\setbeamercolor{alerted text}{bg=mzesbg,fg=mzesgold}
	\setbeamercovered{invisible}
}

% Font
\setbeamerfont{frametitle, frametitle continuation}{size=\large}
\setbeamerfont{titlelike}{size=\large}
\setbeamerfont{footline}{size=\scriptsize}

% Logo
\logo{\includegraphics[width=12em]{mzes-logo-solo-4c.eps}}
\newcommand{\nologo}{\setbeamertemplate{logo}{}}
% customize foot line
\makeatother
\setbeamertemplate{footline}
{
	\leavevmode%
	\hbox{%
		\begin{beamercolorbox}[wd=.25\paperwidth,ht=2.25ex,dp=1ex,center]{author in head/foot}%
			\usebeamerfont{author in head/foot}\insertshortauthor
		\end{beamercolorbox}%
		\begin{beamercolorbox}[wd=.6\paperwidth,ht=2.25ex,dp=1ex,center]{title in head/foot}%
			\usebeamerfont{title in head/foot}\insertshorttitle
		\end{beamercolorbox}%
		\begin{beamercolorbox}[wd=.15\paperwidth,ht=2.25ex,dp=1ex,center]{page in head/foot}%
			\insertframenumber{} / \inserttotalframenumber\hspace*{1ex}
	\end{beamercolorbox}}%
	\vskip0pt%
}
\makeatletter

% Change the width of the progress bar to make it more visible (Code taken from here: https://github.com/matze/mtheme/issues/237)
\makeatletter
\setlength{\metropolis@titleseparator@linewidth}{0.5pt} % Title page
\setlength{\metropolis@progressonsectionpage@linewidth}{1pt} % Progress bar on section page
\setlength{\metropolis@progressinheadfoot@linewidth}{1pt} % Progress bar in header
\makeatother

\setbeamertemplate{navigation symbols}{}
\providecommand*\email[1]{\href{mailto:#1}{#1}}

% title slide
\title[Short Title] % (optional, use only with long paper titles)
{\large \textbf{MZES Beamer Metropolis Theme}} 
\subtitle{\small A Modern MZES Look for \LaTeX-generated PDF Slides}

\author[Cohen \& Co-Author]{%
	\texorpdfstring{
		\begin{columns}
			\begin{column}{0.45\textwidth}
				Denis Cohen \\
				\scriptsize University of Mannheim \vspace{1em} \\
				\tiny \hspace*{1em} \faEnvelope \hspace{.5em} \texttt{\email{denis.cohen@mzes.uni-mannheim.de}} \\
				\tiny \hspace*{1em} \faGlobe \hspace{.5em}  \texttt{\href{https://denis-cohen.github.io/}{denis-cohen.github.io}} \\
				\tiny \hspace*{1em} \faTwitter \hspace{.5em} \texttt{\href{https://twitter.com/denis_cohen}{@denis\_cohen}}
			\end{column}
			\begin{column}{0.45\textwidth}
				Author 2 \\
				\scriptsize Institution 2 \vspace{1em} \\
				\tiny \hspace*{1em} \faEnvelope \hspace{.5em} \texttt{\email{author2@institution2.edu}} \\
				\tiny \hspace*{1em} \faGlobe \hspace{.5em}  \texttt{\href{https://}{author2.com}} \\
				\tiny \hspace*{1em} \faTwitter \hspace{.5em} \texttt{\href{https://twitter.com/}{@author2}}
			\end{column}
		\end{columns}
	}
	{Cohen \& Co-Author}
}
\date{}
\institute{Cool Conference, Cool Place \\ \today}
\subject{}



\begin{document}

\begin{frame}
  \titlepage
\end{frame}

\nologo{ % no logos except on the title page
	\section{Key Facts}
	\begin{frame}{Outline}
		\tableofcontents[currentsection]
	\end{frame}

	\begin{frame}{What is the MZES Beamer Metropolis Theme?}
		\begin{itemize}
			 \item This theme, inspired by the MZES corporate design, is a customization of the popular \href{https://github.com/matze/mtheme}{\texttt{metropolis}} theme for Beamer
			 \item A matching MZES Xaringan Theme is available \href{https://github.com/denis-cohen/mzes-xaringan-metropolis/}{here}
		\end{itemize}
	\end{frame}
	
	\begin{frame}{MZES Xaringan Metropolis vs MZES Beamer Metropolis}
	\small
	\cemph{Similarities}
		\begin{itemize}
			\item Both themes use MZES corporate colors.
			\item Both feature title slides with full contact and social media information for multiple authors.
			\item Both embed the MZES Logo on the first slide.
			\item Both use the 16:9 aspect ratio (instead of 4:3), the default on most projectors and screens nowadays.
			\item Both allow for emphases in \cemph{MZES cyan} and \gemph{MZES gold}.
		\end{itemize}
	\end{frame}

		\begin{frame}{MZES Xaringan Metropolis vs MZES Beamer Metropolis}
	\small
	\cemph{Differences}
		\begin{itemize}
			\item Beamer produces static PDF slides, Xaringan produces dynamic, interactive HTML5 slides.
			\item Beamer slides are written in LaTeX, Xaringan slides are written in RMarkdown.
			\item Unlike Beamer, Xaringan allows for R code to be evaluated while knitting the code and for (interactive) R output to be directly embedded in the presentation.
			\item Unlike the Beamer Theme, the Xaringan Theme does not support a permanent footer.
			\item Unlike the Beamer Theme, the Xaringan Theme does not support progress bars on section title slides. Progress bars are only displayed underneath the slide title.
			\item The Xaringan Theme does not support the Beamer Theme's custom blocks, though equivalent classes can be defined in \texttt{mtheme.css}.
			\item The Xaringan Theme does not support the Beamer Theme's optional custom citation commands.
			\item Whereas Beamer directly supports BibTeX references, Xaringan requires a little \href{https://github.com/yihui/xaringan/wiki/Bibliography-and-citations}{work-around} using the R package \href{https://cran.r-project.org/web/packages/RefManageR/index.html}{\texttt{RefManageR}}.
		\end{itemize}
	\end{frame}
	
	\section{Cool Features}
	\begin{frame}{A Sample Slide}
		A sample list:
		\begin{itemize}
			\item \textcolor{mzesgold}{emphasis 1}
			\item \textcolor{mzescyan}{emphasis 2}
			\item \textcolor{mzesdarkgold}{emphasis 3}
		\end{itemize}
	\end{frame}
	
	\begin{frame}{Some Sample Blocks}
		\begin{block}{Remark}
			Sample text
		\end{block}
		
		\begin{alertblock}{Important theorem}
			Sample text
		\end{alertblock}
	\end{frame}
	
	\begin{frame}{Some Custom Referencing Styles}
	Lorem ipsum dolor sit amet, consectetur adipiscing elit, sed do eiusmod tempor incididunt ut labore et dolore magna aliqua.\cpsh{Downs1957c} 
	
	Ut enim ad minim veniam, quis nostrud exercitation ullamco laboris nisi ut aliquip ex ea commodo consequat.\cpsl{Stokes1963} 
	
	Duis aute irure dolor in reprehenderit in voluptate velit esse cillum dolore eu fugiat nulla pariatur.\cpshl[e.g. ]{KKV, Greene2003, Greene2012}{Wooldridge2002, Gelman2007} 
	
	\end{frame}
	
	\section{Bibliography}
	\begin{frame}[allowframebreaks]{Bibliography}
	\printbibliography[heading=none]
	\end{frame}
}
\end{document}